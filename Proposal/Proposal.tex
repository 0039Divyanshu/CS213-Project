\documentclass{article}
\usepackage{titling}

\begin{document}
\title{CS213 Project Proposal}
\author{
    Anirudh Voruganti
    \\ 190010005
    \and
    Divyanshu Nagrale
    \\ 190010015
    \and
    Harsh Kumar
    \\ 190010019
}
\date{\today}
\maketitle

\pagebreak

\title{Anonymous Doubt Clearing System}
\author{}
\date{}
\maketitle

\section{Problem Statement}
The aim of this project is to streamline the method of
doubt clearing and optimize the system so one can easily
keep track of their previous doubts and revisit them later.

\section{Project Components}
This is a full-stack web-developement project. There will be three classes of people
that can access different types of interfaces catering to
their position.
\begin{description}
    \item[Admin] The admin can list courses, assign teachers and students to those courses.
    \item[Students] The students can ask doubt to instructors that they have taken a course of.
    \item[Instructors] The instructors can answer the students' doubt and also make anouncement to all the students in a particular course that teach.
\end{description}
The communication side of thing will be handled through an
email-like interface. The students can only communicate to
the instructors, while the instructors can communicate with
the students in two different ways.
\begin{itemize}
    \item \textbf{Individually:} To clear individual doubts of students.
    \item \textbf{Course-wise Anouncements:} To make any anouncements to everyone taking a particular course that the instructor takes.
\end{itemize}
The idea is to make a method of communication more formal than 
simple WhatsApp chat, but also less formal that an email. And 
make it anonymous for the students.\\
When a doubt is asked by a student, the instructor can not
see who has asked it, so as to make the identity of the 
person who asked it anonymous. This will enble the students to
freely ask whatever they want.


\end{document}